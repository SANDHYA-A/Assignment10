\documentclass[journal,12pt,twocolumn]{IEEEtran}

\usepackage{setspace}
\usepackage{gensymb}
\usepackage[cmex10]{amsmath}
\usepackage{amsthm}

\usepackage{mathrsfs}
\usepackage{txfonts}
\usepackage{stfloats}
\usepackage{bm}
\usepackage{cite}
\usepackage{cases}
\usepackage{subfig}

\usepackage{longtable}
\usepackage{multirow}

\usepackage{enumitem}
\usepackage{mathtools}
\usepackage{steinmetz}
\usepackage{tikz}
\usepackage{circuitikz}
\usepackage{verbatim}
\usepackage{tfrupee}
\usepackage[breaklinks=true]{hyperref}
\usepackage{graphicx}
\usepackage{tkz-euclide}
\usepackage{afterpage}
\usetikzlibrary{calc,math}
\usepackage{listings}
    \usepackage{color}                                            
    \usepackage{array}                                            
    \usepackage{longtable}                                       
    \usepackage{calc}                                             
    \usepackage{multirow}                                         
    \usepackage{hhline}                                       
    \usepackage{ifthen}                                           
    \usepackage{lscape}     
\usepackage{multicol}
\usepackage{chngcntr}

\DeclareMathOperator*{\Res}{Res}

\renewcommand\thesection{\arabic{section}}
\renewcommand\thesubsection{\thesection.\arabic{subsection}}
\renewcommand\thesubsubsection{\thesubsection.\arabic{subsubsection}}

\renewcommand\thesectiondis{\arabic{section}}
\renewcommand\thesubsectiondis{\thesectiondis.\arabic{subsection}}
\renewcommand\thesubsubsectiondis{\thesubsectiondis.\arabic{subsubsection}}


\hyphenation{op-tical net-works semi-conduc-tor}
\def\inputGnumericTable{}                                 

\lstset{
%language=C,
frame=single, 
breaklines=true,
columns=fullflexible
}
\begin{document}


\newtheorem{theorem}{Theorem}[section]
\newtheorem{problem}{Problem}
\newtheorem{proposition}{Proposition}[section]
\newtheorem{lemma}{Lemma}[section]
\newtheorem{corollary}[theorem]{Corollary}
\newtheorem{example}{Example}[section]
\newtheorem{definition}[problem]{Definition}

\newcommand{\BEQA}{\begin{eqnarray}}
\newcommand{\EEQA}{\end{eqnarray}}
\newcommand{\define}{\stackrel{\triangle}{=}}
\bibliographystyle{IEEEtran}
\providecommand{\mbf}{\mathbf}
\providecommand{\pr}[1]{\ensuremath{\Pr\left(#1\right)}}
\providecommand{\qfunc}[1]{\ensuremath{Q\left(#1\right)}}
\providecommand{\sbrak}[1]{\ensuremath{{}\left[#1\right]}}
\providecommand{\lsbrak}[1]{\ensuremath{{}\left[#1\right.}}
\providecommand{\rsbrak}[1]{\ensuremath{{}\left.#1\right]}}
\providecommand{\brak}[1]{\ensuremath{\left(#1\right)}}
\providecommand{\lbrak}[1]{\ensuremath{\left(#1\right.}}
\providecommand{\rbrak}[1]{\ensuremath{\left.#1\right)}}
\providecommand{\cbrak}[1]{\ensuremath{\left\{#1\right\}}}
\providecommand{\lcbrak}[1]{\ensuremath{\left\{#1\right.}}
\providecommand{\rcbrak}[1]{\ensuremath{\left.#1\right\}}}
\theoremstyle{remark}
\newtheorem{rem}{Remark}
\newcommand{\sgn}{\mathop{\mathrm{sgn}}}
\providecommand{\abs}[1]{\left\vert#1\right\vert}
\providecommand{\res}[1]{\Res\displaylimits_{#1}} 
\providecommand{\norm}[1]{\left\lVert#1\right\rVert}
%\providecommand{\norm}[1]{\lVert#1\rVert}
\providecommand{\mtx}[1]{\mathbf{#1}}
\providecommand{\mean}[1]{E\left[ #1 \right]}
\providecommand{\fourier}{\overset{\mathcal{F}}{ \rightleftharpoons}}
%\providecommand{\hilbert}{\overset{\mathcal{H}}{ \rightleftharpoons}}
\providecommand{\system}{\overset{\mathcal{H}}{ \longleftrightarrow}}
	%\newcommand{\solution}[2]{\textbf{Solution:}{#1}}
\newcommand{\solution}{\noindent \textbf{Solution: }}
\newcommand{\cosec}{\,\text{cosec}\,}
\providecommand{\dec}[2]{\ensuremath{\overset{#1}{\underset{#2}{\gtrless}}}}
\newcommand{\myvec}[1]{\ensuremath{\begin{pmatrix}#1\end{pmatrix}}}
\newcommand{\mydet}[1]{\ensuremath{\begin{vmatrix}#1\end{vmatrix}}}
\numberwithin{equation}{subsection}
\makeatletter
\@addtoreset{figure}{problem}
\makeatother
\let\StandardTheFigure\thefigure
\let\vec\mathbf
\renewcommand{\thefigure}{\theproblem}
\def\putbox#1#2#3{\makebox[0in][l]{\makebox[#1][l]{}\raisebox{\baselineskip}[0in][0in]{\raisebox{#2}[0in][0in]{#3}}}}
     \def\rightbox#1{\makebox[0in][r]{#1}}
     \def\centbox#1{\makebox[0in]{#1}}
     \def\topbox#1{\raisebox{-\baselineskip}[0in][0in]{#1}}
     \def\midbox#1{\raisebox{-0.5\baselineskip}[0in][0in]{#1}}
\vspace{3cm}
\title{Matrix Theory EE5609 - Assignment 10\\
}

\author{\IEEEauthorblockN{Sandhya Addetla}\\
\IEEEauthorblockA{PhD Artificial Inteligence Department} \\
AI20RESCH14001\\
 }

\maketitle
\begin{abstract}
Find rank of the matrix.
\end{abstract}
Download  python code from 
\begin{lstlisting}
https://github.com/SANDHYA-A/Assignment10
\end{lstlisting}
\section{Problem}
Let $J$ denote the matrix of order $n \times n$ with all
entries 1 and let $B$ be a $3n \times 3n$ matrix given by
$B = \myvec{0&0&J\\0&J&0\\J&0&0}$. \\
Find rank of matrix $B$.

\section{Solution}
\begin{table}[h!]
\begin{center}
\begin{tabular}{ | m{3cm} | m{5cm}| } \hline 
%\begin{tabular}{|c|c|}\hline
Given  &  a) Matrix $J$ of $n \times n$ dimension with all entries 1.\\&  
b) Matrix $B$ of $3n \times 3n$ dimension { 
\begin{align*}
B = \myvec{0&0&J\\0&J&0\\J&0&0}
\end{align*}}\\  \hline
Transforming matrix $B$ into Block diagonal matrix using transformation Matrix & {\begin{align*}
M = \vec{T}(B)\\
M = \myvec{0&0&I\\0&I&0\\I&0&0} \myvec{0&0&J\\0&J&0\\J&0&0}\\
M = \myvec{J&0&0\\0&J&0\\0&0&J}
\end{align*}}\\  \hline
Rank of Block Diagonal matrix $M$ & It is equal to the sum of rank of individual blocks in diagonal{\begin{align*}
r(J) = 1\\
\therefore r(M) = 1 + 1 + 1 = 3
\end{align*}}\\ \hline
Rank of a matrix and its transformation are same. & $\therefore$ rank of matrix $B$ is {\begin{align*}r(B) = r(M) =3
\end{align*}}\\ \hline 
\end{tabular}
\end{center}
\end{table}

\begin{table}[h!]
\begin{center}
\begin{tabular}{ | m{5cm} | m{10cm}| } \hline 
%\begin{tabular}{|c|c|}\hline
Example &  Let $n=2$ \\&  
{ \begin{align*}
J = \myvec{1&1\\1&1}\\
B = \myvec{0&0&0&0&1&1\\0&0&0&0&1&1\\0&0&1&1&0&0\\0&0&1&1&0&0\\1&1&0&0&0&0\\1&1&0&0&0&0}
\end{align*}}\\  \hline
Transforming matrix $B$ into Block diagonal matrix using transformation Matrix & {\begin{align*}
M = \vec{T}(B)\\
M =  \myvec{0&0&0&0&1&0\\0&0&0&0&0&1\\0&0&1&0&0&0\\0&0&0&1&0&0\\1&0&0&0&0&0\\0&1&0&0&0&0} \myvec{0&0&0&0&1&1\\0&0&0&0&1&1\\0&0&1&1&0&0\\0&0&1&1&0&0\\1&1&0&0&0&0\\1&1&0&0&0&0}\\
M = \myvec{1&1&0&0&0&0\\1&1&0&0&0&0\\0&0&1&1&0&0\\0&0&1&1&0&0\\0&0&0&0&1&1\\0&0&0&0&1&1}
\end{align*}}\\  \hline
Rank of Block Diagonal matrix $M$ & It is equal to the sum of rank of individual blocks in diagonal{\begin{align*}
r(J) = 1\\
\therefore r(M) = 1 + 1 + 1 = 3
\end{align*}}\\ \hline
Rank of a matrix and its transformation are same. & $\therefore$ rank of matrix $B$ is {\begin{align*}r(B) = r(M) =3
\end{align*}}\\ \hline 
\end{tabular}
\end{center}
\end{table}
\pagebreak
\afterpage{
\subsection*{Approach 2:} 
\begin{center}
\begin{tabular}{ | m{3cm} | m{15cm}| } \hline 
%\begin{tabular}{|c|c|}\hline
Example &  Let $n=2$ \\&  
{ \begin{align*}
J = \myvec{1&1\\1&1}\\
B = \myvec{0&0&0&0&1&1\\0&0&0&0&1&1\\0&0&1&1&0&0\\0&0&1&1&0&0\\1&1&0&0&0&0\\1&1&0&0&0&0}
\end{align*}}\\  \hline
Representing matrix $B$ as Jordan normal form & {\begin{align*}
B = \vec{S}(N)\vec{S}^{-1}\\
B=\myvec{-1&0&0&0&1&-1\\1&0&0&0&1&-1\\0&-1&0&1&0&0\\0&1&0&1&0&0\\0&0&-1&0&1&1\\0&0&-1&0&1&1} \myvec{0&0&0&0&0&0\\0&0&0&0&0&0\\0&0&0&0&0&0\\0&0&0&2&0&0\\0&0&0&0&2&0\\0&0&0&0&0&-2}
\myvec{\frac{-1}{2}&\frac{1}{2}&0&0&0&0\\0&0&\frac{-1}{2}&\frac{1}{2}&0&0\\0&0&0&0&\frac{-1}{2}&\frac{1}{2}\\0&0&\frac{1}{2}&\frac{1}{2}&0&0\\\frac{1}{4}&\frac{1}{4}&0&0&\frac{1}{4}&\frac{1}{4}\\\frac{-1}{4}&\frac{-1}{4}&0&0&\frac{1}{4}&\frac{1}{4}}
\end{align*}}\\  \hline
The matrix $N$ in Jordan Normal form is &{\begin{align*}
\myvec{0&0&0&0&0&0\\0&0&0&0&0&0\\0&0&0&0&0&0\\0&0&0&2&0&0\\0&0&0&0&2&0\\0&0&0&0&0&-2}
\end{align*}}\\ \hline
Rank of Block Diagonal matrix $N$ & It is equal to the sum of rank of individual blocks in diagonal{\begin{align*}
\therefore r(N) = 3
\end{align*}}\\ \hline
Rank of a matrix $B$ and its Jordan Normal form  are same. & $\therefore$ rank of matrix $B$ is {\begin{align*}r(B) = r(N) =3
\end{align*}}\\ \hline 
\end{tabular}
\end{center}
}

\end{document}