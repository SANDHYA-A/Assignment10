\documentclass[journal,12pt,twocolumn]{IEEEtran}
\IEEEoverridecommandlockouts
\usepackage{cite}
\usepackage{amsmath,amssymb,amsfonts,bm}
\usepackage{mathtools}
\usepackage{tkz-euclide} 
\usepackage{tikz}
\usetikzlibrary{calc,math}
 \usepackage{caption}
\usepackage{listings}
\usepackage{gensymb}
\let\vec\mathbf
\numberwithin{equation}{subsection}

\newcommand{\myvec}[1]{\ensuremath{\begin{pmatrix}#1\end{pmatrix}}}
\newcommand{\norm}[1]{\left\lVert#1\right\rVert}
\newcommand{\mydet}[1]{\ensuremath{\begin{vmatrix}#1\end{vmatrix}}}

\renewcommand\thesection{\arabic{section}}
\renewcommand\thesubsection{\thesection.\arabic{subsection}}
\renewcommand\thesubsubsection{\thesubsection.\arabic{subsubsection}}

\renewcommand\thesectiondis{\arabic{section}}
\renewcommand\thesubsectiondis{\thesectiondis.\arabic{subsection}}
\renewcommand\thesubsubsectiondis{\thesubsectiondis.\arabic{subsubsection}}
%\renewcommand{\theequation}{\theenumi}
%\numberwithin{equation}{enumi}

\providecommand{\mbf}{\mathbf}
\providecommand{\pr}[1]{\ensuremath{\Pr\left(#1\right)}}
\providecommand{\qfunc}[1]{\ensuremath{Q\left(#1\right)}}
\providecommand{\sbrak}[1]{\ensuremath{{}\left[#1\right]}}
\providecommand{\lsbrak}[1]{\ensuremath{{}\left[#1\right.}}
\providecommand{\rsbrak}[1]{\ensuremath{{}\left.#1\right]}}
\providecommand{\brak}[1]{\ensuremath{\left(#1\right)}}
\providecommand{\lbrak}[1]{\ensuremath{\left(#1\right.}}
\providecommand{\rbrak}[1]{\ensuremath{\left.#1\right)}}
\providecommand{\cbrak}[1]{\ensuremath{\left\{#1\right\}}}
\providecommand{\lcbrak}[1]{\ensuremath{\left\{#1\right.}}
\providecommand{\rcbrak}[1]{\ensuremath{\left.#1\right\}}}

\lstset{
frame=single, 
breaklines=true,
columns=fullflexible
}

\begin{document}

\title{Matrix Theory EE5609 - Assignment 10\\
}

\author{\IEEEauthorblockN{Sandhya Addetla}\\
\IEEEauthorblockA{PhD Artificial Inteligence Department} \\
AI20RESCH14001\\
 }

\maketitle
\begin{abstract}
Find rank of the matrix.
\end{abstract}
Download  python code from 
\begin{lstlisting}
https://github.com/SANDHYA-A/Assignment10
\end{lstlisting}
\section{Problem}
Let $J$ denote the matrix of order $n \times n$ with all
entries 1 and let $B$ be a $3n \times 3n$ matrix given by
$B = \myvec{0&0&J\\0&J&0\\J&0&0}$. \\
Find rank of matrix B.
\section{Solution}
Let $J$ be an $ n \times n$ with all entries 1. Matrix $B$ is given as:-\\
$B = \myvec{0&0&J\\0&J&0\\J&0&0}$\\
The rank of a matrix is given by the number of linearly independent row vectors in the matrix.We can observe that $n$ rows of matrix $B$, formed by matrix $J$ are identical, hence there are three linearly independent row vectors in matrix $B$. \\
$\therefore$ the rank of matrix $B$ is 3.\\
Example: 
\begin{align}
\text{ Let, } n =3\\
J = \myvec{1&1&1\\1&1&1\\1&1&1}\\
B=\myvec{0&0&J\\0&J&0\\J&0&0}\\
=\myvec{0&0&0&0&0&0&1&1&1\\0&0&0&0&0&0&1&1&1\\0&0&0&0&0&0&1&1&1\\0&0&0&1&1&1&0&0&0\\0&0&0&1&1&1&0&0&0\\0&0&0&1&1&1&0&0&0\\1&1&1&0&0&0&0&0&0\\1&1&1&0&0&0&0&0&0\\1&1&1&0&0&0&0&0&0}
\end{align}
We can observe that there are three linearly independent row vectors in the above matrix.\\
Hence, rank of the matrix $B$ is 3.
\end{document}